\Chapter{CD-melléklet tartalma}

Dolgozatomhoz egy darab CD-melléklet tartozik, melynek tartalma a következő:

\bigskip

\noindent \texttt{Dolgozat} katalógus:

\begin{itemize}
\item \texttt{GTFS-alapu\_menetrendnyilvantarto-rendszer.pdf}: \\ a dolgozatot tartalmazó fájl, PDF-formátumban.
\item \texttt{kiiras.pdf}: A feladatkiírást tartalmazó fájl, PDF-formátumban.
\item \texttt{osszefoglalas.pdf}: Magyar nyelvű összefoglaló, PDF-formátumban.
\item \texttt{osszefoglalas.tex}: Magyar nyelvű összefoglaló, \LaTeX-formátumban.
\item \texttt{summary.pdf}: Angol nyelvű összefoglaló, PDF-formátumban.
\item \texttt{summary.tex}: Angol nyelvű összefoglaló, \LaTeX-formátumban.
\end{itemize}

\bigskip

\noindent \texttt{LaTeX} katalógus:

a dolgozat \LaTeX-kódját tartalmazza.

\bigskip

\noindent \texttt{Forraskod} katalógus:

Az alkalmazás forráskódját tartalmazza.
