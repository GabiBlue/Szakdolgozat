\documentclass[a4paper,12pt]{article}

% Set margins
\usepackage[hmargin=3cm, vmargin=3cm]{geometry}

\frenchspacing

% Language packages
\usepackage[utf8]{inputenc}
\usepackage[T1]{fontenc}
\usepackage[magyar]{babel}

% AMS
\usepackage{amssymb,amsmath}

% Graphic packages
\usepackage{graphicx}

% Colors
\usepackage{color}
\usepackage[usenames,dvipsnames]{xcolor}

% Enumeration
\usepackage{enumitem}

% Links
\usepackage{hyperref}

\linespread{1.2}

\begin{document}

\pagestyle{empty}

\section*{Összegzés}

\textit{Javora Gábor: GTFS-alapú menetrend-nyilvántartó rendszer}

\bigskip

A szakdolgozatom célja egy olyan webalkalmazás készítése volt, melyben a menetrendadatok a nemzetközi szabvánnyá vált GTFS-formátumban vannak tárolva, a menetrend böngészése könnyen átlátható formában valósul meg a felhasználók számára, amit útvonaltervező funkció is segít. Az alkalmazás elkészítéséhez kliensoldalon AngularJS-t, szerveroldalon pedig Python / Flask keretrendszert használtam. Sikerült egy olyan alkalmazást készíteni, amely a GTFSDB library segítségével bármelyik GTFS-t használó közlekedési vállalat menetrendi adataival kompatibilis.

Az elkészített alkalmazás már a jelenlegi állapotában is használható, a megvalósítani kívánt funkciók egytől egyig működőképesek. Vannak azonban ötleteim a későbbi továbbfejlesztési lehetőségekre, néhányat ismertetnék ezek közül. Véleményem szerint leginkább az útvonaltervező funkciót lehetne tovább finomítani, kibővíteni még több lehetőséggel. Ilyen lehetne például a következő ajánlott útvonal számítása az aktuális idő, kiindulópont és a célállomás ismeretében. Hasznos lehetne az átszállások számát minimalizáló opciót, vagy a várakozási idők minimalizálását implementálni, amit a felhasználó ki tudna választani, hogy milyen szempont szerinti optimális útvonaltervet szeretne. Szóba jöhetne egy limit az átszállásra vonatkozóan, például több legyen, mint három perc, hogy biztosan elegendő idő legyen az átszállásra. A menetrend böngészése kapcsán továbbfejlesztési lehetőségként az egyes viszonylatok menetrendjének a nyomtatható formában történő mentésének a támogatása is egy hasznos ötletnek tűnik. Nagyon érdekes lehet még az ismert megállók, és az azokat összekötő útvonalak esetében kiszámolni egy lehetséges – valamilyen szempontból optimális – menetrendet, ami tehát a keresési problémának a megfordítása lenne.

\end{document}

