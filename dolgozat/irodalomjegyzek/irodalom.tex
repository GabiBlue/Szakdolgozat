\begin{thebibliography}{x}
\addcontentsline{toc}{chapter}{\bibname}

\bibitem{python} https://hu.wikipedia.org/wiki/Python\_(programoz%C3%A1si\_nyelv) 

\bibitem{flask} 

\bibitem{sqlalchemy} https://hu.wikipedia.org/wiki/SQLAlchemy

\bibitem{html} https://hu.wikipedia.org/wiki/HTML

\bibitem{javascrip} https://developer.mozilla.org/hu/docs/Web/JavaScript

\bibitem{angularjs} http://nyelvek.inf.elte.hu/leirasok/JavaScript/index.php?chapter=27

\bibitem{css} https://hu.wikipedia.org/wiki/Cascading\_Style\_Sheets

\bibitem{bootstrap} http://nora707.tryfruit.com/2016/10/25/bootstrap/

\bibitem{git} https://into.hu/hirek/csapatmunka-git-el-1-resz-mi-is-az-a-git

\bibitem{github}

\bibitem{gtfs}
https://developers.google.com/transit/gtfs/

\bibitem{gtfsspec}
https://en.wikipedia.org/wiki/General\_Transit\_Feed\_Specification

\bibitem{grafabrazolas}
http://tamop412.elte.hu/tananyagok/algoritmusok/lecke23\_lap1.html

\bibitem{legrovidebbut}
https://web.cs.elte.hu/blobs/diplomamunkak/bsc\_matelem/2009/podobni\_katalin.pdf

\bibitem{dijkstra}
https://hu.wikipedia.org/wiki/Dijkstra-algoritmus

\bibitem{floyd-warshall}
https://web.cs.elte.hu/blobs/diplomamunkak/bsc\_matelem/2009/podobni\_katalin.pdf

\end{thebibliography}
