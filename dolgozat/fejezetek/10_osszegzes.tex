\Chapter{Összegzés}

%* Ezt elég lesz majd csak a végén megcsinálni.
%* Ebben már csak át kell majd tekinteni az elkészült dolgokat.
%* 1-2 oldalnál (a bevezetéshez hasonlóan) ennek sem kell hosszabbnak lennie.

A szakdolgozatom célja egy olyan webalkalmazás készítése volt, melyben a menetrendadatok a nemzetközi szabvánnyá vált GTFS-formátumban vannak tárolva, a menetrend böngészése könnyen átlátható formában valósul meg a felhasználók számára, amit útvonaltervező funkció is segít. Az alkalmazás elkészítéséhez kliensoldalon AngularJS-t, szerveroldalon pedig Python / Flask keretrendszert használtam. Sikerült egy olyan alkalmazást készíteni, amely a GTFSDB library segítségével bármelyik GTFS-t használó közlekedési vállalat menetrendi adataival kompatibilis.

Az elkészített alkalmazás már a jelenlegi állapotában is használható, a megvalósítani kívánt funkciók egytől egyig működőképesek. Vannak azonban ötleteim a későbbi továbbfejlesztési lehetőségekre, néhányat ismertetnék ezek közül. Véleményem szerint leginkább az útvonaltervező funkciót lehetne tovább finomítani, kibővíteni még több lehetőséggel. Ilyen lehetne például a következő ajánlott útvonal számítása az aktuális idő, kiindulópont és a célállomás ismeretében. Hasznos lehetne az átszállások számát minimalizáló opciót, vagy a várakozási idők minimalizálását implementálni, amit a felhasználó ki tudna választani, hogy milyen szempont szerinti optimális útvonaltervet szeretne. Szóba jöhetne egy limit az átszállásra vonatkozóan, például több legyen, mint három perc, hogy biztosan elegendő idő legyen az átszállásra. A menetrend böngészése kapcsán továbbfejlesztési lehetőségként az egyes viszonylatok menetrendjének a nyomtatható formában történő mentésének a támogatása is egy hasznos ötletnek tűnik. Nagyon érdekes lehet még az ismert megállók, és az azokat összekötő útvonalak esetében kiszámolni egy lehetséges – valamilyen szempontból optimális – menetrendet, ami tehát a keresési problémának a megfordítása lenne.

\Chapter{Summary}

The goal of my thesis work was to create a web application which stores the schedule information in GTFS format. This format has become an international standard. The browsing of the schedule information is easy via the web interface. It also provides functions for trip planning. I used AngularJS on the client side and Python / Flask framework on the server side. I developed an application with the GTFSDB library, which is compatible with the timetable of any transit agency using GTFS.

The created application is ready to use in its current state. Each of the planned functions have been implemented. However, I have some ideas for further development. I would like to mention some of these. In my opinion, the trip planning function could be refined by adding more options. For example, it can calculate the following recommended route based on the current time, a starting point, and a destination. It could be useful to minimize the number of transfers or waiting times. The user would be able to choose,the optimization criteria for finding the appropriate route. There could be a limit on the transfer, for example more than three minutes is enough time to transfer. As a further option for browsing the timetable, it could be implemented a function for export the timetable to a printable format. It may be very interesting to calculate an optimum timetable from some points of view based on known stops and routes connecting them. In fact, it is the inverse of the search problem.
