\Chapter{A back end megvalósítása}

Szerveroldalon egy többrétegű struktúrát alakítottam ki, mint azt már említettem az alkalmazás felépítésének a részletezésénél. A fejezetben a két felső réteg, a Flask-alkalmazás és a menetrend csomag kerül bemutatásra.

\Section{Flask-webalkalmazás}

% Azt kellene bemutatni, hogy az menetrend lib funkcióit hogy sikerült kivezetni, hogy REST API-n keresztül elérhető legyen.
Ez a réteg dolgozza fel a klienstől érkező kéréseket a menetrend csomagra építve. Ha vannak a kérésnek paraméterei, akkor először azoknak a helyességét vizsgálja. Számos esetben ezt egy, a menetrend csomagban lévő, függvény segítségével végzi. A válaszokhoz az adatokat szintén a menetrend csomag egyes függvényeinek a meghívásával állítja elő. Az így előállt eredményadatokat JSON-formátumban küldi a kliens felé.
A réteg a \texttt{server.py} fájlban van megvalósítva. A benne található függvények \texttt{route()} dekorátorokkal vannak ellátva. Ezek írják le, hogy mely URL-re hívódjon meg az adott függvény. Az URL-ek és a függvények nevei az egyszerűség kedvéért megegyeznek.
A függvényeket és leírásukat tartalmazza a következő táblázat.

\begin{longtable}{|p{4.5cm}|p{9cm}|}
\hline
\textbf{URL} & \textbf{Leírás} \\
\hline
\texttt{/} &
Az alkalmazás megnyitásakor vagy a főoldal gombra történő kattintáskor elküldi az \texttt{index.html} fájlt. \\
\hline
\texttt{/favicon.ico} &
A favicon megjelenítésére szolgál. A favicon a weboldalhoz tartozó ikon, amely a böngészőben megjelenhet a címsorban, a fül elején és a könyvjelzők előtt. \\
\hline
\texttt{/get\_bus\_routes} &
Az autóbuszvonalakat adja vissza. A lekérdezéshez a \texttt{query\_routes\_by\_route\_type()} függvényt hívja meg. A paraméter adja meg a típust, a 3-as jelenti az autóbuszt. \\
\hline
\texttt{/get\_tram\_routes} &
A villamosvonalakat adja vissza. A fent említett függvényt hívja meg, 0-t adva paraméterként, hiszen ez jelenti a villamostípust. \\
\hline
\texttt{/get\_start\_times} &
Egy viszonylat egy irányának egy adott napi indulási idejeit adja vissza.
Paraméterek:
\begin{itemize}
\item \texttt{route\_short\_name}: a viszonylatot határozza meg
\item \texttt{direction\_id}: az irányt határozza meg
\item \texttt{date}: a dátumot határozza meg
\end{itemize}
A függvény először a paraméterek helyességét vizsgálja. Ha a \texttt{direction\_id} értéke nem 0 vagy 1, vagy ha egy járat sem található a megadott \texttt{route\_short\_name} és \texttt{direction\_id} értékekkel, akkor 404-es hibakóddal tér vissza. Ennek az az oka, hogy ez normál használat során nem fordulhat elő, csak ha kézzel át lett írva az URL.
Ha helyesek a paraméterek, akkor a \texttt{query\_trips\_schedule()} függvény segítségével lekérdezi a paramétereknek megfelelő járatokat. Ha egyet sem talál, akkor az adott napon nem közlekedik a viszonylat. Ellenkező esetben a \texttt{generate\_schedule\_dictionary()}  függvénnyel elkészíti az aznapi indulási időket tartalmazó dictionary-t. \\
\hline
\texttt{/get\_dates} &
Az adatbázisban a menetrendi adatokhoz tartozó dátumokat adja vissza. Ezekből a dátumokból lehet választani a menetrendböngésző oldalon. A lekérdezéshez a \texttt{query\_dates()} függvényt használja.  \\
\hline
\texttt{/get\_trip\_stops} &
Egy járat részletes menetrendjét adja vissza. Paraméterként egy \texttt{trip\_id}-t kap, először teszteli, hogy valós-e. Ha nem, 400-as hibakóddal tér vissza. Ha igen, akkor a \texttt{query\_trip\_stops()} függvénnyel lekérdezi a járat menetrendjét.  \\
\hline
\texttt{/get\_trip\_shapes} &
Egy járat útvonalát adja vissza. Paraméterként egy \texttt{trip\_id}-t kap, először teszteli, hogy valós-e. Ha nem, 400-as hibakóddal tér vissza. Ha igen, akkor a \texttt{query\_trip\_shapes()} függvénnyel lekérdezi a járat útvonalát. \\
\hline
\texttt{/get\_trip\_terminals} \texttt{\_coordinates} &
Egy járat induló és végállomásának a koordinátáit adja vissza. Paraméterként egy \texttt{trip\_id}-t kap, először teszteli, hogy valós-e. Ha nem, 400-as hibakóddal tér vissza. Ha igen, akkor a \texttt{query\_trip\_terminals\_coordinates()} függvénnyel lekérdezi az említett koordinátákat. \\
\hline
\texttt{/get\_trip\_stops\_} \texttt{coordinates} &
Egy járat megállóinak a koordinátáit adja vissza, kivéve az indulóét és a végállomásét. Paraméterként egy \texttt{trip\_id}-t kap, először teszteli, hogy valós-e. Ha nem, 400-as hibakóddal tér vissza. Ha igen, akkor a \texttt{query\_trip\_stops\_coordinates()} függvénnyel lekérdezi az említett koordinátákat. \\
\hline
\texttt{/get\_all\_stops} &
Az adatbázisban található összes megálló nevét adja vissza. A lekérdezéshez a \texttt{query\_all\_stops()} függvényt használja. \\
\hline
\texttt{/plan\_trip} &
A megadott induló és célmegálló közt tervezett útvonalat adja vissza. Az útvonaltervezéshez a \texttt{dijkstra()} függvényt használja. \\
\hline
\end{longtable}

\Section{A menetrend library}

A csomagban található függvényekkel lehet az adatbázisból adatokat lekérdezni, az SQLAlchemy és a GTFSDB segítségével. A következő táblázat a csomagban lévő fájlokat és az azokban lévő függvényeket ismerteti.

\begin{longtable}{|p{4.5cm}|p{9cm}|}
\hline
\textbf{Fájlnév} & \textbf{Függvények leírásai} \\
\hline
\texttt{graph.py} &
\texttt{generate\_graph()}:

Az adatbázisban található megállókból egy súlyozott, irányított gráfot épít fel, az alapján, hogy honnan hova és mennyi idő alatt lehet eljutni. A függvény működése a hetedik fejezetben részletesen ki van fejtve.
\\
\hline
\texttt{routes.py} &
\texttt{query\_routes\_by\_route\_type(route\_type)}:

Az adatbázisból a paraméterként megadott típusú viszonylatokat kérdezi le, dictionary-k listájaként visszaadva az eredményt.

\bigskip

\texttt{test\_route\_short\_name(route\_short\_name, direction\_id)}:

Megvizsgálja, hogy a paraméterként megadott viszonylatnév létezik-e az adatbázisban. Ha igen, akkor visszatér vele.
\\
\hline
\texttt{schedule.py} &
\texttt{query\_dates()}:

Az adatbázisban a menetrendi adatokhoz tartozó dátumokat adja vissza egy listában.

\bigskip

\texttt{query\_trips\_schedule(route\_short\_name, direction\_id, date, trip)}:

A paramétereknek megfelelő járatokat és a járatok adatait tartalmazó dictionary-t adja vissza.

\bigskip

\texttt{generate\_schedule\_dictionary(trips, trips\_dict)}:

A paraméterként kapott járatok indulási időivel kibővíti a paraméterként átadott dictionary-t, majd ezzel a kibővített dictionary-vel tér vissza.
\\
\hline
\texttt{stops.py} &
\texttt{query\_all\_stops()}:

Az adatbázisban található összes megálló nevét adja vissza egy listában.
\\
\hline
\texttt{trip\_planner.py} &
\texttt{dijkstra(graph, src, dest)}:

Dijkstra-algoritmussal a paraméterként átadott gráfban való legrövidebb utat keresi meg két, szintén paraméterként átadott, megálló között. Az algoritmus a hetedik fejezetben kerül részletes bemutatásra.
\\
\hline
\texttt{trips.py} &
\texttt{test\_trip(trip\_id)}:

Megvizsgálja, hogy a paraméterként kapott azonosítójú járat létezik-e az adatbázisban. Ha létezik, visszatér vele.

\bigskip

\texttt{query\_trip\_stops(trip):}

Egy dictionary-ben visszaadja a paraméterként kapott járat nevét, induló és végállomását, valamint a dictionary-n belül egy listában a sorrendben érintett megállókat, indulási idővel és az adott megállóig eltelt menetidővel.

\bigskip

\texttt{query\_trip\_shapes(trip)}:

A paraméterként megadott járat útvonalát adja vissza egy listában.

\bigskip

\texttt{query\_trip\_terminals\_coordinates(trip)}:

A paraméterként megadott járat induló és végállomásának a koordinátáit adja vissza egy listában.

\bigskip

\texttt{query\_trip\_stops\_coordinates(trip)}:

Egy listában a paraméterként megadott járat megállóinak a koordinátáit adja vissza, kivéve az indulóét és a végállomásét.
\\
\hline
\end{longtable}

A menetrend csomagnak számos előnye van véleményem szerint. Egyrészt ennek a rétegnek a bevezetésével jobban el tudtam különíteni az egyes funkciókat. A flaskos réteg csak a kérések helyességének vizsgálatával és a válaszok generálásával foglalkozik. A lekérdezések megfogalmazása és az eredmények feldolgozása a menetrend csomag függvényeiben történik, ezáltal átláthatóbb is lett a Flask-fájl, hiszen rövidebbek az egyes függvények törzsei.

A legnagyobb előny, véleményem szerint, a jövőbeli továbbfejleszthetőség nagymértékű megkönnyítésében rejlik. Ha később úgy döntök (vagy bárki más úgy dönt), hogy a MEnetrend webalkalmazásnak el kívánom (kívánja) készíteni a mobilos vagy asztali változatát, akkor a back enden csak a Flaskot kell lecserélni, az alsóbb rétegekben megvalósított funkciók ugyanazok maradhatnak. Esetleg egy másik, hasonló tematikájú webalkalmazás is építkezhet a menetrend csomagban megvalósított néhány funkcióra.

% A menetrend library, mint wrapper a GTFSDB felé

% Az előnye, hogy így más kliensekkel is lehet használni (mobil, desktop app, vagy másik webapp).
