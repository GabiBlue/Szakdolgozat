\Chapter{Tesztelés}

Az elkészített alkalmazást kipróbáltam több különböző közlekedési társaság GTFS-adatbázisával is. Ehhez az adatokat a \url{https://transitfeeds.com} oldalról szereztem be. Az oldal nyilvános GTFS-adatbázisok archívuma, elsősorban szoftverfejlesztők számára, a világ minden tájáról összegyűjtve. Jelenleg több mint 500 különböző város menetrendjének az archívuma található az oldalon. Ezek közül választottam ki hatot, amelyekkel teszteltem az alkalmazásomat. Négy magyarországi (Budapest, Miskolc, Szeged, Pécs) és két külföldi (Bécs, Firenze) várost választottam.

Megvizsgáltam, hogy a nyers adatokból mennyi idő alatt jön létre az adatbázis a GTFSDB \texttt{database\_load} függvényével. Az eredményeket összesíti \aref{tab:gtfs}. táblázat.

\begin{table}
\centering
\begin{tabular}{|l|r|r|r|r|}
\hline
\textbf{Menetrend} & \textbf{ZIP fájlméret} & \textbf{trips} $\quad$ & \textbf{stop\_times} & \textbf{Betöltési idő} \\
& (MB) & (darabszám) & (darabszám) & (másodperc) \\
\hline
Bécs & 14,6 & 120 181 & 2 169 529 & 1 395 \\
\hline
Budapest & 39,3 & 227 900 & 4 518 698 & 2 945 \\
\hline
Firenze & 10 & 40 683 & 1 017 698 & 836 \\
\hline
Miskolc & 1,2 & 9 641 & 148 785 & 97 \\
\hline
Szeged & 1,9 & 17 162 & 302 816 & 191 \\
\hline
Pécs & 1,5 & 5 563 & 111 192 & 71 \\
\hline
\end{tabular}
\caption{GTFS adatbázisok mennyiségi jellemzői és betöltési ideje}
\label{tab:gtfs}
\end{table}

A táblázatban látható a letöltött adatfájl mérete, a betöltési idő másodpercben, illetve hogy a \texttt{trips} és a \texttt{stop\_times} táblák hány darab rekordból állnak. Azért ezt a kettőt emeltem ki, mert legnagyobb részt ezek határozzák meg, hogy mennyi lesz a betöltési idő. Kisebb városok esetén, ahol néhány százezer rekordról van szó, viszonylag hamar, 1-3 perc alatt fel lehet tölteni az adatbázist. Nagyobb, több millió rekordból álló adathalmaz esetén ez már több 10 percig is eltart. Általában egy GTFS feed egy hónapra előre tartalmazza a menetrendet, ezért ezt a műveletet ritkán kell elvégezni. Mivel általában nem változik hónapról hónapra drasztikusan a menetrend, ezért elegendő lehet csak a \texttt{calendar} és a \texttt{calendar\_dates} táblák aktualizálása a következő havi dátumokkal, valamint az esetleges megváltozott menetrendű járatok frissítése, az adatbázis legnagyobb részét elegendő egyszer feltölteni, és az utána hosszú hónapokig aktuális marad.

Készítettem egy statisztikát, hogy a \texttt{generate\_graph} függvény mennyi idő alatt fut le, vagyis az adott hálózat megállóiból mennyi idő alatt építi fel az útvonalkereséshez a gráfot. Ugyanazokat az adatbázisokat használtam, amiket az előző teszthez összegyűjtöttem. Ezt foglalja össze \aref{tab:runtime}. táblázat.

\begin{table}
\centering
\begin{tabular}{|l|r|r|}
\hline
\textbf{Menetrend} & \textbf{Megállók száma} & \textbf{generate\_graph} \textbf{futási ideje} \\
& & (másodperc) \\
\hline
Bécs & 4 357 & 17,3 \\
\hline
Budapest & 7 391 & 28,7 \\
\hline
Firenze & 2 307 & 10,3 \\
\hline
Miskolc & 561 & 3,1 \\
\hline
Szeged & 574 & 2,9 \\
\hline
Pécs & 541 & 2,6 \\
\hline
\end{tabular}
\caption{Gráf generálásának futási ideje a futási idő függvényében}
\label{tab:runtime}
\end{table}

\newpage

Hasonlóan az adatbázisok betöltési idejéhez, a gráf felépítésének ideje is az adatbázis méretével arányos. Ebben az esetben a megállók száma a mérvadó, hiszen a készítendő gráfnak ugyanannyi csomópontja lesz, illetve az élek számával is arányos. A megfelelő felhasználói élmény biztosítása érdekében a gráfot egyszer kell felépíteni, és az eredményt gyorsítótárban kell tárolni.
