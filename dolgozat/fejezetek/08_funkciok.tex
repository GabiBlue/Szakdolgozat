\Chapter{További funkciók}

Külön fejezetekbe kerülhetnek a további funkciók, vagy komplikáltabb megoldások. Csak ötlet szintjén:
- Következő ajánlott útvonal számítása az aktuális idő, kiindulópont és a célállomás ismeretében. (Az útvonalkereső is gyakorlatilag ezt csinálja, itt annyi a különbség, hogy kvázi kiírja a felhasználónak, hogy merre érdemes indulnia, és vált, ha közben megváltozik a javaslat az eltelt idő miatt.)
- Menetrend mentése valamilyen nyomtatható formában.
- Minimalizálás az átszállások számára vonatkozóan.
- Kieső járattal kapcsolatos számítások.
- Várakozási idők minimalizálása, vagy limit az átszállásra vonatkozóan (például több legyen, mint 3 perc, hogy biztos legyen idő átszállni).