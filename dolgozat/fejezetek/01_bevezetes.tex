\Chapter{Bevezetés}

%* Ezt elég lesz majd csak a végén összerakni.
%* Itt arról kell meggyőzni az olvasót, hogy milyen jó lesz neki, ha végigolvassa az egész dolgozatot. :)
%* Majd az elkészült eredményekből látszik, de tetszetős lehet majd itt hangoztatni, hogy mennyire rugalmas, szabványoknak megfelelő, és %átvihető megoldásról van szó.
A helyi menetrendek adatai nyilvánosan elérhetők. Számos web- és mobilalkalmazást is szoktak biztosítani az egyes közlekedési vállalatok, viszont ezek használata gyakran nehézkes, vagy legalábbis nem illeszkedik a felhasználók tényleges elvárásaihoz. A dolgozatom célja egy olyan webalkalmazás készítése, melyben a menetrendadatok a nemzetközi szabvánnyá vált GTFS-formátumban vannak tárolva, illetve a felhasználók számára a kívánt információk könnyebben, jobban átlátható formában elérhetővé válnak, amit útvonaltervező funkció is segít.

A GTFS (General Transit Feed Specification) egy Google által kifejlesztett, nyilvános és ingyenes formátum, melyben tetszőleges tömegközlekedési hálózat adatai eltárolhatók, földrajzi pozíciókkal együtt. Ez lehetőséget ad a járatok és a megállók térképes megjelenítésére, valamint a menetrendalapú útvonaltervezésre is. Számos közlekedési vállalat – többek közt az MVK Zrt. is – elérhetővé teszi GTFS-adatbázisukat programozók, fejlesztők részére, ezáltal lehetővé téve, hogy ezen adatokra építve szoftvereket fejlesszenek.

Az alkalmazás kliensoldalon AngularJS-t, szerveroldalon pedig Python/Flask keretrendszert fog használni. Az adatbázist pedig a GTFS-formátum szerint fogom kialakítani, azzal a céllal, hogy bármelyik GTFS-t használó közlekedési vállalat menetrendi adataival kompatibilis legyen.

Szakdolgozatom első részében a menetrendi adatok nyilvántartásának módjait és néhány hasonló alkalmazást szeretnék bemutatni. Ezt követően a használni kívánt technológiákat és az alkalmazás főbb részeit ismertetem. Részletesen bemutatom a GTFS-adatbázisokat, majd az alkalmazás szerver- és kliensoldali megvalósítását. Kitérek a gráfábrázolás és az útvonalkeresés lehetséges módjaira, ismertetve az általam használt megoldásokat. Végezetül az elkészített alkalmazás tesztelését tekintem át.
