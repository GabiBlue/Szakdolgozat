\documentclass[a4paper,12pt]{article}

% Set margins
\usepackage[hmargin=3cm, vmargin=3cm]{geometry}

\frenchspacing

% Language packages
\usepackage[utf8]{inputenc}
\usepackage[T1]{fontenc}
\usepackage[magyar]{babel}

% AMS
\usepackage{amssymb,amsmath}

% Graphic packages
\usepackage{graphicx}

% Colors
\usepackage{color}
\usepackage[usenames,dvipsnames]{xcolor}

% Enumeration
\usepackage{enumitem}

% Links
\usepackage{hyperref}

\linespread{1.2}

\begin{document}

\pagestyle{empty}

\section*{Summary}

\textit{Gábor Javora: GTFS-based Public Transport Schedule Application}

\bigskip

The goal of my thesis work was to create a web application which stores the schedule information in GTFS format. This format has become an international standard. The browsing of the schedule information is easy via the web interface. It also provides functions for trip planning. I used AngularJS on the client side and Python / Flask framework on the server side. I developed an application with the GTFSDB library, which is compatible with the timetable of any transit agency using GTFS.

The created application is ready to use in its current state. Each of the planned functions has been implemented. However, I have some ideas for further development. I would like to mention some of these. In my opinion, the trip planning function could be refined by adding more options. For example, it can calculate the following recommended route based on the current time, a starting point, and a destination. It could be useful to minimize the number of transfers or waiting times. The user would be able to choose the optimization criteria for finding the appropriate route. There could be a limit on the transfer, for example more than three minutes is enough time to transfer. As a further option for browsing the timetable, it could be implemented a function for export the timetable to a printable format. It may be very interesting to calculate an optimum timetable from some points of view based on known stops and routes connecting them. In fact, it is the inverse of the search problem.

\end{document}

